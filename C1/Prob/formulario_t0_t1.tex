% Documento hecho a partir de la plantilla de LaTeX: https://github.com/Manuelbelgicano/DGIIM/blob/master/extra/plantilla_apuntes_v2.tex

\documentclass[11pt,twoside,titlepage,a4paper]{article}

%%%%%%%%%%%%%%%%%%%%%%%%%%%%%%%%%%%%%%%%%
%			   COLORINES				%
%%%%%%%%%%%%%%%%%%%%%%%%%%%%%%%%%%%%%%%%%
\usepackage{xcolor}
\definecolor{rojooscuro}{HTML}{8A0808}
\definecolor{burdeos}{HTML}{610B0B}
\definecolor{rojomorado}{HTML}{B40431}

%%%%%%%%%%%%%%%%%%%%%%%%%%%%%%%%%%%%%%%%%
%			  MATEMÁTICAS				%
%%%%%%%%%%%%%%%%%%%%%%%%%%%%%%%%%%%%%%%%%
\usepackage{amsmath} % Matemáticas
\usepackage{amsfonts} % Letras caligráficas para matemáticas
\usepackage{mathtools} % Matemáticas extra
\usepackage{amssymb} % Símbolos extra
\usepackage{amsthm} % Teoremas
% Eliminar '*' para añadir numeración a los lemas, teoremas...
\theoremstyle{definition}
\newtheorem*{defi}{Definición} % Comando para las definiciones
\newtheoremstyle{plain_rojo}{}{}{}{}{\color{rojooscuro}\bfseries}{:}{ }{}
\theoremstyle{plain_rojo}
\newtheorem*{lem}{Lema} % Comando para los lemas
\newtheorem*{teo}{Teorema} % Comando para los teoremas
\theoremstyle{remark}
\newtheorem*{cor}{Corolario} % Comando para los corolarios
\renewenvironment{proof}{{\bfseries\color{rojooscuro}Demostración:}}{\qed} % Cambiar el título de las demostraciones

%%%%%%%%%%%%%%%%%%%%%%%%%%%%%%%%%%%%%%%%%
%			   TIPOGRAFÍA				%
%%%%%%%%%%%%%%%%%%%%%%%%%%%%%%%%%%%%%%%%%
\usepackage{CrimsonPro}
\usepackage[T1]{fontenc}
% The font package uses mweights.sty which has som issues with the
% \normalfont command. The following two lines fixes this issue.
\let\oldnormalfont\normalfont
\def\normalfont{\oldnormalfont\mdseries}

%%%%%%%%%%%%%%%%%%%%%%%%%%%%%%%%%%%%%%%%%
%				  CÓDIGO				%
%%%%%%%%%%%%%%%%%%%%%%%%%%%%%%%%%%%%%%%%%
\usepackage{listingsutf8}
\lstset{
	inputencoding=utf8/latin1, % Codificación
	xleftmargin=1em, % Margen extra a la izquierda
	breaklines=true, % Romper líneas largas
	language=C++, % Lenguaje del código
	frame=single, % Enmarcado
	numbers=left, % Números de línea
	numbersep=8pt, % Separación de los números de línea
	tabsize=4, % Tamaño de los tabs
	frame=leftline, % Posición del enmarcado
	framerule=2pt, % Grosor del enmarcado
	showstringspaces=false, % Mostrar los espacios en las cadenas de caracteres
	basicstyle=\footnotesize\ttfamily, % Estilo del código
	keywordstyle=\color{burdeos}, % Estilo de las palabras reservadas
	numberstyle=\normalfont, % Estilo de los números de línea
	rulecolor=\color{rojooscuro}, % Estilo del enmarcado
	commentstyle=\color{red}, % Estilo de los comentarios
	stringstyle=\color{rojomorado} % Estilo de las cadenas de caracteres
}

%%%%%%%%%%%%%%%%%%%%%%%%%%%%%%%%%%%%%%%%%
%				MÁRGENES				%
%%%%%%%%%%%%%%%%%%%%%%%%%%%%%%%%%%%%%%%%%
\usepackage[a4paper]{geometry}
\geometry{
	left=2.5cm, % Margen izquierdo
	right=2.5cm, % Margen derecho
	bottom=2.5cm % Margen inferior
}

%%%%%%%%%%%%%%%%%%%%%%%%%%%%%%%%%%%%%%%%%
%  			  LISTAS/TABLAS				%
%%%%%%%%%%%%%%%%%%%%%%%%%%%%%%%%%%%%%%%%%
\usepackage{enumitem} % Opciones de personalización de listas
\renewcommand{\arraystretch}{1.3} %Cambiar el tamaño entre líneas de una tabla

%%%%%%%%%%%%%%%%%%%%%%%%%%%%%%%%%%%%%%%%%
%		COMANDOS PERSONALIZADOS 		%
%%%%%%%%%%%%%%%%%%%%%%%%%%%%%%%%%%%%%%%%%
\newcommand{\autores}{ % Autores del documento
	\begin{tabular}{l}
	Manuel Gachs Ballegeer
	\end{tabular}
}
\newcommand{\institucion}{ % Insitución
	Universidad de Granada
}
\newcommand{\infoextra}{ % Año o cualquier otra información para el título
	curso 2019-2020
}

%%%%%%%%%%%%%%%%%%%%%%%%%%%%%%%%%%%%%%%%%
%		ENCABEZADO/PIE DE PAGINA		%
%%%%%%%%%%%%%%%%%%%%%%%%%%%%%%%%%%%%%%%%%
\usepackage{fancyhdr}
\setlength{\headheight}{14pt}
\pagestyle{fancy}
\fancyhf{}
% Para que aparezca el título de la sección y no el número 
\renewcommand{\sectionmark}[1]{%
\markboth{#1}{}}
% Encabezado
\fancyhead[LE,RO]{\color{burdeos}{\leftmark}} % A la izquierda en pares, derecha en impares
\fancyhead[RE,LO]{\color{burdeos}{\institucion}} % A la derecha en pares, izquierda en impares
% Pie de página
\fancyfoot[LE,RO]{\Large\textbf{\thepage}} % A la izquierda en pares, derecha en impares
\renewcommand{\headrulewidth}{0.5pt} % Grosor de la línea

%%%%%%%%%%%%%%%%%%%%%%%%%%%%%%%%%%%%%%%%%
%			   	TÍTULOS					%
%%%%%%%%%%%%%%%%%%%%%%%%%%%%%%%%%%%%%%%%%
\usepackage{titlesec}
\titleformat{\section} % Estilo de las secciones
{\color{rojooscuro}\Huge\bfseries}
{\color{rojooscuro}\thesection}{1em}{}

%%%%%%%%%%%%%%%%%%%%%%%%%%%%%%%%%%%%%%%%%
%		   	  MISCELÁNEO				%
%%%%%%%%%%%%%%%%%%%%%%%%%%%%%%%%%%%%%%%%%
\usepackage{pagecolor} % Colorear las portadas
\renewcommand{\contentsname}{Índice} % Cambiar el título del índice
\setlength\parindent{0pt} % Tamaño de la sangría
\usepackage{graphicx} % Imágenes
\usepackage{blindtext} % Texto de relleno (Se puede eliminar)
\usepackage{multicol} % Para poner las listas por columnas

\begin{document}

%%%%%%%%%%%%%%%%%%%%%%%%%%%%%%%%%%%%%%%%%
%				 PORTADA 				%
%%%%%%%%%%%%%%%%%%%%%%%%%%%%%%%%%%%%%%%%%
\begin{titlepage}
	\newpagecolor{rojooscuro} % Color de la portada
	\parbox[t]{\textwidth}{
		\raggedright
		\color{white}{\LARGE{\textbf{\institucion}}} \\
		\textit{\infoextra}
	}
	\vfill
	\parbox[c]{\textwidth}{
		\color{white}{
			\fontsize{70pt}{70pt}{\textbf{Probabilidad}} \\
			\bigskip \\
			\fontsize{40pt}{40pt}{\emph{Formulario}}
		}
	}
	\vfill
	\parbox[t]{\textwidth}{
		\raggedright
		\color{white}{\Large{\autores}} \\
	}
\end{titlepage}
\restorepagecolor

%%%%%%%%%%%%%%%%%%%%%%%%%%%%%%%%%%%%%%%%%
%				 ÍNDICE 				%
%%%%%%%%%%%%%%%%%%%%%%%%%%%%%%%%%%%%%%%%%
\tableofcontents
\clearpage

%%%%%%%%%%%%%%%%%%%%%%%%%%%%%%%%%%%%%%%%%
%				 DOCUMENTO 				%
%%%%%%%%%%%%%%%%%%%%%%%%%%%%%%%%%%%%%%%%%

%%%%%%%%%%%%%%%%%%%%%%%%%%%%%%%%%%%%%%%%%
%  	   ESPACIOS DE PROBABILIDAD			%
%%%%%%%%%%%%%%%%%%%%%%%%%%%%%%%%%%%%%%%%%

\section{Espacios de probabilidad}

Propiedades de la probabilidad:
\begin{itemize}[noitemsep]
	\item $P(\emptyset)=0,\;P(\Omega)=1$
	\item $P(\bar{A})=1-P(A)\;\;\forall A\in\mathcal{A}$
	\item $P(\bigcup_{i=1}^{n}A_i)=\sum_{i=1}^{n}P(A_i)$ donde $A_i\cap A_j=\emptyset,\;\forall j\neq i$
	\item $P(\bigcup_{i=1}^{n}A_i)\leq\sum_{i=1}^{n}P(A_i)$
	\item Si $A\subseteq B$, entonces $P(A)\leq P(B)$
	\item $P(A\cup B)=P(A)+P(B)-P(A\cap B)$
	\item $P(A\cup B\cup C)=P(A)+P(B)+P(C)-P(A\cap B)-P(A\cap C)-P(B\cap C)+P(A\cap B\cap C)$
	\item $P(A-B)=P(A\cap\bar{B})=P(A)-P(A\cap B)$
	\item Leyes de De Morgan:
	\begin{itemize}
		\item $\overline{A\cup B}=\bar{A}\cap\bar{B}$
		\item $\overline{A\cap B}=\bar{A}\cup\bar{B}$
	\end{itemize}
	\item $P(\bigcap_{i=1}^{n}A_i)\geq 1-\sum_{i=1}^{n}P(\bar{A}_i)$ (Desigualdad de Boole)
\end{itemize}

\subsubsection*{Probabilidad condicionada}

Cálculo de la probabilidad condicionada:
$$P(B/A)=\displaystyle\frac{P(A\cap B)}{P(A)}$$
En caso de que los sucesos $A$ y $B$ sean independientes, $P(A/B)=P(A)$.

Cálculo de la probabilidad condicionada a la intersección de sucesos:
$$P(C/A\cap B)=\displaystyle\frac{P(A\cap B\cap C)}{P(A)P(B/A)}$$
Cálculo de la intersección de sucesos a partir de las probabilidades condicionadas:
$$P(A_1\cap A_2\cap\ldots\cap A_n)=P(A_1)P(A_2/A_1)P(A_3/A_2\cap A_1)\cdots P(A_n/A_{n-1}\cap\ldots\cap A_1)$$
Si los sucesos son independientes, entonces $P(A_1\cap A_2\cap\ldots\cap A_n)=P(A_1)P(A_2)\cdots P(A_n)$

\subsubsection*{Teorema de la probabilidad total}

$$P(B)=\displaystyle\sum_{i=1}^{n}P(A_1)P(B/A_i)$$

\subsubsection*{Teorema de Bayes}

$$P(A_i/B)=\displaystyle\frac{P(A_i)P(B/A_i)}{\sum_{i=1}^{n}P(A_i)P(B/A_i)}$$

%%%%%%%%%%%%%%%%%%%%%%%%%%%%%%%%%%%%%%%%%
%		 DISTRIBUCIONES DISCRETAS		%
%%%%%%%%%%%%%%%%%%%%%%%%%%%%%%%%%%%%%%%%%

\newpage
\section{Distribuciones discretas}

\subsection{Distribución degenerada $X\rightsquigarrow D(c)$}

Función masa de probabilidad:
\begin{equation*}
P[X=x]=\left\{
	\begin{array}{l c l}
	1 & si & x=c \\
	0 & si & x\neq c
	\end{array}
\right.
\end{equation*}
Función de distribución:
\begin{equation*}
F(x)=\left\{
	\begin{array}{l c l}
	0 & si & x<c \\
	1 & si & x\geq c
	\end{array}
\right.
\end{equation*}
Función generatriz de momentos: $M_x(t)=E[e^{tx}]=e^{tc}$, $\forall t\in\mathbb{R}$

\subsection{Distribución uniforme discreta $X\rightsquigarrow U\{x_1,\ldots,x_n\}$}

Función masa de probabilidad:
\begin{equation*}
P[X=x]=\left\{
	\begin{array}{l c l}
	^1/_n & si & x\in\{x_1,\ldots,x_n\} \\
	0 & si & x\notin\{x_1,\ldots,x_n\}
	\end{array}
\right.
\end{equation*}
Función de distribución:
\begin{equation*}
F(x)=\left\{
	\begin{array}{l c l}
	0 & si & x<x_1 \\
	^1/_n & si & x_1\leq x<x_2 \\
	\vdots && \\
	^{n-1}/_n & si & x_{n-1}\leq x<x_n \\
	1 & si & x\geq x_n
	\end{array}
\right.
\end{equation*}
Función generatriz de momentos: $M_x(t)=E[e^{tx}]=\frac 1n\sum e^{tx_i}$, $\forall t\in\mathbb{R}$
\begin{multicols}{2}
	\begin{itemize}[label={}]
		\item $E[X]=\frac 1n\sum x_i$
		\item $m_k=\frac 1n\sum x_i^k$
		\item $Var[X]=\frac 1n\sum x_i^2-(\frac 1n\sum x_i)^2$
		\item $\mu_k=\frac 1n\sum(x_i-\mu)^k$
	\end{itemize}
\end{multicols}

\subsection{Distribución de Bernouilli $X\rightsquigarrow B(p)$}

Función masa de probabilidad:
\begin{equation*}
P[X=x]=\left\{
	\begin{array}{l c l}
	p & si & x=1 \\
	q=1-p & si & x=0
	\end{array}
\right.
\end{equation*}
Función de distribución:
\begin{equation*}
F(x)=\left\{
	\begin{array}{l c l}
	0 & si & x<0 \\
	q & si & 0\leq x<1 \\
	1 & si & x\geq 1
	\end{array}
\right.
\end{equation*}
Función generatriz de momentos: $M_x(t)=E[e^{tx}]=pe^t+q$, $\forall t\in\mathbb{R}$
\begin{multicols}{2}
	\begin{itemize}[label={}]
		\item $E[X]=p$
		\item $m_k=p^k$
		\item $Var[X]=pq$
		\item $\mu_k=q^kp-p^kq$
	\end{itemize}
\end{multicols}

\subsection{Distribución binomial $X\rightsquigarrow B(n,p)$}

Función masa de probabilidad:
\begin{equation*}
P[X=x]=\left(
	\!\begin{array}{c}
	n \\
	x
	\end{array}
\!\right)p^xq^{n-x},\quad x\in\{0,1,\ldots,n\}
\end{equation*}
Función generatriz de momentos: $M_x(t)=E[e^{tx}]=(pe^t+(1-p))^n$, $\forall t\in\mathbb{R}$
\begin{multicols}{2}
	\begin{itemize}[label={}]
		\item $E[X]=np$
		\item $Var[X]=npq$
	\end{itemize}
\end{multicols}

\subsection{Distribución geométrica $X\rightsquigarrow G(p)$}

\begin{equation*}
\underbrace{\begin{array}{c c}
	\text{\emph{Número de fracasos antes del primer éxito}} & \text{\emph{Número de intentos antes del primer éxito}} \\
	P[X=x]=q^xp & P[X=x]=q^{x-1}p \\
	M_{x_1}(t)=\displaystyle\frac{p}{1-qe^t},\quad\forall t<-\ln q 
	& M_{x_2}(t)=\displaystyle\frac{p}{e^t-q},\quad\forall t>\ln q \\
	E_1[X]=\displaystyle\frac qp & E_2[X]=\displaystyle\frac 1p \\
	\end{array}}
\end{equation*}
$$Var[X]=\displaystyle\frac{q}{p^2}$$

\subsection{Distribución binomial negativa $X\rightsquigarrow BN(r,p)$}

\begin{equation*}
\underbrace{\begin{array}{c c}
	\text{\emph{Número de fracasos antes del r-ésimo éxito}} & \text{\emph{Número de pruebas antes del r-ésimo éxito}} \\
	P[X=x]=\left(\!\begin{array}{c}
	x+r-1 \\
	x
	\end{array}\!\right)q^xp^r,\;\; x=0,1,\ldots
	 & P[X=x]=\left(\!\begin{array}{c}
	x-1 \\
	r-1
	\end{array}\!\right)q^{x-r}p^r,\;\; x=1,2,\ldots \\
	M_{x_1}(t)=\Big(\displaystyle\frac{p}{1-qe^t}\Big)^r,\quad\forall t<-\ln q 
	& M_{x_2}(t)=\Big(\displaystyle\frac{p}{e^t-q}\Big)^r,\quad\forall t>\ln q \\
	E_1[X]=\displaystyle\frac{rq}{p} & E_2[X]=\displaystyle\frac rp \\
	\end{array}}
\end{equation*}
$$Var[X]=\displaystyle\frac{rq}{p^2}$$

\subsection{Distribución de Poisson $X\rightsquigarrow P(\lambda)$}

Función masa de probabilidad:
\begin{equation*}
P[X=x]=e^{-\lambda}\frac{\lambda^x}{x!},\quad x\in[0,+\infty[
\end{equation*}
Función generatriz de momentos: $M_x(t)=e^{-\lambda(e^t-1)}$, $\forall t\in\mathbb{R}$
$$E[X]=Var[X]=\lambda$$

\subsection{Distribución hipergeométrica $X\rightsquigarrow H(N,n,k)$}

Función masa de probabilidad:
\begin{equation*}
P[X=x]=\frac{
			\left(\!\begin{array}{c}
			k \\
			x
			\end{array}
			\!\right)
			\left(\!\begin{array}{c}
			N-k \\
			n-x
			\end{array}
			\!\right)}{
			\left(\!\begin{array}{c}
			N \\
			n
			\end{array}
			\!\right)}
\end{equation*}
\begin{multicols}{2}
	\begin{itemize}[label={}]
		\item $E[X]=k\displaystyle\frac nN$
		\item $Var[X]=\displaystyle\frac{kn}{N}(1-\frac kN)\frac{N-n}{N-1}$
	\end{itemize}
\end{multicols}

%%%%%%%%%%%%%%%%%%%%%%%%%%%%%%%%%%%%%%%%%
%		 DISTRIBUCIONES CONTINUAS		%
%%%%%%%%%%%%%%%%%%%%%%%%%%%%%%%%%%%%%%%%%

\section{Distribuciones continuas}

\subsection{Distribución uniforme continua $X\rightsquigarrow U(a,b)$}

Función de densidad:
\begin{equation*}
f(x)=\left\{
	\begin{array}{l c l}
	\displaystyle\frac{1}{b-a} & si & x\in]a,b[\\
	0 & si & x\notin]a,b[
	\end{array}
\right.
\end{equation*}
Función de distribución:
\begin{equation*}
F(x)=\left\{
	\begin{array}{l c l}
	0 & si & x<a \\
	\displaystyle\frac{x-a}{b-a} & si & a\leq x<b \\
	1 & si & x\geq b
	\end{array}
\right.
\end{equation*}
Función generatriz de momentos: $M_x(t)=E[e^{tx}]=\frac 1n\sum e^{tx_i}$, $\forall t\in\mathbb{R}$
\begin{multicols}{2}
	\begin{itemize}[label={}]
		\item $E[X]=\displaystyle\frac{a+b}{2}$\quad(punto medio del intervalo)
		\item $Var[X]=\displaystyle\frac{(b-a)^2}{12}$
	\end{itemize}
\end{multicols}

\subsection{Distribución normal $X\rightsquigarrow N(\mu,\sigma^2)$}

Función de densidad:
$$f(x)=\frac{1}{\sigma\sqrt{2\pi}}e^{-\displaystyle\frac{(x-\mu)^2}{2\sigma^2}}$$

\subsubsection{Tipificación}

Teniendo $X\rightsquigarrow N(\mu,\sigma^2)$, $Z=\displaystyle\frac{X-\mu}{\sigma}$, entonces:
\begin{enumerate}[font={\color{rojooscuro}\bfseries}]
	\item $Z\rightsquigarrow N(0,1)$
	\item $F_X(x)=F_Z\Big(\displaystyle\frac{x-\mu}{\sigma}\Big)$
\end{enumerate}

\subsubsection{Aproximaciones a la normal}

\begin{itemize}[font={\color{rojooscuro}\bfseries}]
	\item \textbf{Aproximación de la binomial: } Se hace cuando $n>10$ y $np>5$, $nq>5$
	$$X\rightsquigarrow B(n,p)\xrightarrow[X\approx Y]{n\to\infty} Y\rightsquigarrow N(np,npq)$$
	\item \textbf{Aproximación de la Poisson: } Se hace cuando $\lambda>10$
	$$X\rightsquigarrow P(\lambda)\xrightarrow[X\approx Y]{\lambda\to\infty} Y\rightsquigarrow N(\lambda,\lambda)$$
\end{itemize}

\subsection{Distribución exponencial $X\rightsquigarrow exp(\lambda)$}

Función de densidad:
\begin{equation*}
f(x)=\lambda e^{-\lambda x}
\end{equation*}
\begin{multicols}{2}
	\begin{itemize}[label={}]
		\item $E[X]=\displaystyle\frac 1\lambda$
		\item $Var[X]=\displaystyle\frac{1}{\lambda^2}$
	\end{itemize}
\end{multicols}

\subsection{Distribución gamma $X\rightsquigarrow \Gamma(u,\lambda)$}

Función de densidad:
\begin{equation*}
f(x)=\frac{\lambda^u}{\Gamma(u)}x^{u-1}e^{-\lambda x}
\end{equation*}
Valores de la función Gamma: $\Gamma(k)=(k-1)!$

\subsection{Distribución beta $X\rightsquigarrow \beta(p,q)$}

Función de densidad:
\begin{equation*}
f(x)=\frac{1}{\beta(p,q)}x^{p-1}(1-p)^{q-1}
\end{equation*}
Valores de la función beta: $\beta(p,q)=\displaystyle\frac{\Gamma(p)\Gamma(q)}{\Gamma(p+q)}$
\begin{multicols}{2}
	\begin{itemize}[label={}]
		\item $E[X]=\displaystyle\frac{p}{p+q}$
		\item $Var[X]=\displaystyle\frac{pq}{(p+q)^2(p+q+1)}$
	\end{itemize}
\end{multicols}
En el caso de que $p=q=1$, entonces $\beta(p,q)=U(0,1)$.

\end{document}